% Tesi D.S.I. - modello preso da
% Stanford University PhD thesis style -- modifications to the report style
\documentclass[a4paper,12pt]{report}
\usepackage[a4paper]{geometry}
\usepackage{amssymb,amsmath,amsthm}
\usepackage{graphicx}
\usepackage{url}
\usepackage{hyperref}
\usepackage{epsfig}
\usepackage[italian]{babel}
\usepackage{setspace}
\usepackage{tesi}
\usepackage[utf8]{inputenc}
\newtheorem{myteor}{Teorema}[section]
\newenvironment{teor}{\begin{myteor}\sl}{\end{myteor}}
%
%
%			TITOLO: Furthest Insertion Algorithm
%
\begin{document}
\title{Furthest Insertion Algorithm}
\author{Asaf COHEN}
\dept{Corso di Laurea in Informatica} 
\anno{2023-2024}
\matricola{975599}
\relatore{Prof. Giovanni RIGHINI}
%
%        \submitdate{month year in which submitted to GPO}
%		- date LaTeX'd if omitted
%	\copyrightyear{year degree conferred (next year if submitted in Dec.)}
%		- year LaTeX'd (or next year, in December) if omitted
%	\copyrighttrue or \copyrightfalse
%		- produce or don't produce a copyright page (false by default)
%	\figurespagetrue or \figurespagefalse
%		- produce or don't produce a List of Figures page
%		  (false by default)
%	\tablespagetrue or \tablespagefalse
%		- produce or don't produce a List of Tables page
%		  (false by default)
% 
%			DEDICA
%

\beforepreface

\clearpage
\null
\thispagestyle{empty}
\clearpage

\prefacesection{Ringraziamenti}
        {\hfill \Large {\sl dedicato a DA COMPLETARE\dots}}
% 
%			PREFAZIONE
%

\clearpage
\null
\thispagestyle{empty}
\clearpage

\prefacesection{Abstract}
Il problema del commesso viaggiatore (Traveling Salesman Problem o TSP) è un noto problema matematico nel campo dell'ottimizzazione combinatoria, appartiene alla categoria dei problemi NP-difficili, per questo motivo il tempo necessario per trovare la soluzione ottima aumenta esponenzialmente rispetto al numero di città presenti nell'istanza considerata. Nella pratica quindi si possono trovare soluzioni esatte in tempi ragionevoli solo nel caso di istanze piccole (con poche città), nel caso di istanze grandi è possibile considerare algoritmi euristici che non garantiscono l'ottimo, ma producono di fatto soluzioni "accettabili" in poco tempo.
\newline
L'obbiettivo della tesi è progettare un nuovo algoritmo chiamato Furthest Insertion, implementare in un linguaggio di programmazione l'algoritmo, implementare altre importanti euristiche presenti in letteratura (Random Insertion, Nearest Insertion, ...) e analizzare i risultati ottenuti su alcune istanze da TSP-LIB (una nota libreria di istanze TSP).

\afterpreface
% 
% 
%			CAPITOLO 1: Introduzione
\chapter{Introduzione}


\begin{center}
        \begin{tabular}{|c|c|}
                \hline
                \textbf{Numero città} & \textbf{Numero percorsi validi} \\ % Intestazione in grassetto
                \hline % Linea orizzontale sopra della tabella
                4 & 24 \\
                \hline
                5 & 120 \\
                \hline
                6 & 720 \\
                \hline
                7 & 5,040 \\
                \hline
                8 & 40,320 \\
                \hline
                9 & 362,880 \\
                \hline
                10 & 3,628,800 \\
                \hline
                11 & 39,916,800 \\
                \hline
                12 & 479,001,600 \\
                \hline
                13 & 6,227,020,800 \\
                \hline
                14 & 87,178,291,200 \\
                \hline
                15 & 1,307,674,368,000 \\
                \hline
                16 & 20,922,789,888,000 \\
                \hline
                17 & 355,687,428,096,000 \\
                \hline
                18 & 6,402,373,705,728,000 \\
                \hline
                19 & 121,645,100,408,832,000 \\
                \hline
                20 & 2,432,902,008,176,640,000 \\
                \hline
                21 & 51,090,942,171,709,440,000 \\
                \hline
                22 & 1,124,000,727,777,607,680,000 \\
                \hline
                23 & 25,852,016,738,884,976,640,000 \\
                \hline
                24 & 620,448,401,733,239,439,360,000 \\
                \hline
                25 & 15,511,210,043,330,985,984,000,000 \\
                \hline
        \end{tabular}
        \newline
        % Aggiungi una riga per la descrizione sotto alla tabella
        Tabella 1: Numero percorsi validi in relazione al numero di città
\end{center}
%
%

%
%			BIBLIOGRAFIA
%
\begin{thebibliography}{00}
%
\bibitem{gotti91}
M. Gotti, I linguaggi specialistici, Firenze, La Nuova Italia, 1991.
%
\bibitem{wellek62}
R. Wellek, A. Warren, Theory of Literature , 3rd edition, New York, Harcourt, 1962.
%
\bibitem{canziani78}
A. Canziani et al., Come comunica il teatro: dal testo alla scena. Milano, Il Formichiere, 1978.
%
\bibitem{MoD67}
Ministry of Defence, Great Britain, Author and Subject Catalogues of the Naval Library, London, Ministry of Defence, HMSO, 1967.
%
\bibitem{heine23}
H. Heine, Pensieri e ghiribizzi. A cura di A. Meozzi. Lanciano, Carabba, 1923.
%
\bibitem{basso62}
L. Basso, ``Capitalismo monopolistico e strategia operaia'', Problemi del socialismo, vol. 8, n. 5, pp. 585-612, 1962.
%
\bibitem{avirovic93}
L. Avirovic, J. Dodds (a cura di), Atti del Convegno internazionale "Umberto Eco, Claudio Magris. Autori e traduttori a confronto" ( Trieste, 27-28 novembre 1989), Udine, Campanotto, 1993.
%
\bibitem{gans67}
E.L. Gans, "The Discovery of Illusion: Flaubert's Early Works, 1835-1837", unpublished Ph.D. Dissertation, Johns Hopkins University, 1967.
%
\bibitem{harrison92}
R. Harrison, Bibliography of planned languages (excluding Esperanto).  \url{http://www.vor.nu/langlab/bibliog.html}, 1992, agg. 1997.
%
\end{thebibliography}
% 
\end{document}


 
