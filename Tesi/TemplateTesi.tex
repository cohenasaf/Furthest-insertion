%
% Tesi D.S.I. - modello preso da
% Stanford University PhD thesis style -- modifications to the report style
%
%%%%%%%%%%%%%%%%%%%%%%%%%%%%%%%%%%%%%%%%%%%%%%%%%%%%%%%%%%%%%%%%%%%%%%%%%%%
%                                                                         %
%			TESI DOTTORATO                                                   %
%			______________                                                   %
%                                                                         %
%			AUTORE: Elena Pagani                                             %
%                                                                         %
%			Ultima revisione: 7.X.1998                                       %
%           correzioni atrent                                             %
%%%%%%%%%%%%%%%%%%%%%%%%%%%%%%%%%%%%%%%%%%%%%%%%%%%%%%%%%%%%%%%%%%%%%%%%%%%
%
%
\documentclass[a4paper,12pt]{report}
%    \renewcommand{\baselinestretch}{1.6}      % interline spacing
%
% \includeonly{}
%
%			PREAMBOLO
%
\usepackage[a4paper]{geometry}
\usepackage{amssymb,amsmath,amsthm}
\usepackage{graphicx}
\usepackage{url}
\usepackage{hyperref}
\usepackage{epsfig}
\usepackage[italian]{babel}
\usepackage{setspace}
\usepackage{tesi}

% per le accentate
\usepackage[utf8]{inputenc}
%
\newtheorem{myteor}{Teorema}[section]
%
\newenvironment{teor}{\begin{myteor}\sl}{\end{myteor}}
%
%
%			TITOLO
%
\begin{document}
\title{Furthest Insertion Algorithm}
\author{Asaf COHEN}
\dept{Corso di Laurea in Informatica} 
\anno{2023-2024}
\matricola{975599}
\relatore{Prof. Giovanni RIGHINI}
%
%        \submitdate{month year in which submitted to GPO}
%		- date LaTeX'd if omitted
%	\copyrightyear{year degree conferred (next year if submitted in Dec.)}
%		- year LaTeX'd (or next year, in December) if omitted
%	\copyrighttrue or \copyrightfalse
%		- produce or don't produce a copyright page (false by default)
%	\figurespagetrue or \figurespagefalse
%		- produce or don't produce a List of Figures page
%		  (false by default)
%	\tablespagetrue or \tablespagefalse
%		- produce or don't produce a List of Tables page
%		  (false by default)
% 
%			DEDICA
%

\beforepreface

\clearpage
\null
\thispagestyle{empty}
\clearpage

\prefacesection{Ringraziamenti}
        {\hfill \Large {\sl dedicato a DA COMPLETARE\dots}}
% 
%			PREFAZIONE
%

\clearpage
\null
\thispagestyle{empty}
\clearpage

\prefacesection{Abstract}
Il problema del commesso viaggiatore (Traveling Salesman Problem o TSP) è un noto problema matematico nel campo dell'ottimizzazione combinatoria.

Un modo intuitivo per descrivere il problema può essere sintetizzato con la seguente domanda: "Date alcune città, qual è il percorso più breve che visita ogni città esattamente una volta e ritorna alla città di partenza?". \newline \newline
Supponiamo di avere un insieme di città e le distanze tra ogni coppia di città, il problema consiste nel trovare il percorso che attraversa tutte le città una sola volta
e ritorna alla città di partenza, minimizzando la lunghezza totale del percorso.
Un primo approccio (algoritmo esatto) potrebbe consistere nell'enumerare tutti i possibili percorsi e poi procedere nel selezionare il migliore. In questo caso il numero totale di percorsi da analizzare aumenta notevolmente con l'aumentare del numero di città: come illustrato nella tabella della pagina successiva, il numero di percorsi da analizzare è pari a $n!$, quindi se sono presenti n città è necessario confrontare (nel caso peggiore) $n!$ percorsi alla ricerca del migliore, si dice quindi che l'algoritmo esatto (che procede per l'enumerazione totale dei possibili percorsi) ha complessità computazionale $O(n!)$. Questo è il motivo per cui trovare l'ottimo tramite l'enumerazione totale è di fatto impraticabile nei problemi reali dove ci sono migliaia (se non decine di migliaia) di nodi. \newline \newline
Con il tempo sono state sviluppate euristiche, ovvero algoritmi alternativi alla risoluzione esatta (esaustiva), con l'obbiettivo di ottenere soluzioni comunque "buone" ma in tempi veloci, l'obbiettivo di questa tesi consiste nell'implementare una nuova euristica chiamata Furthest Insertion (appartenente alla categoria delle euristiche basate su inserzione) e confrontare i risultati ottenuti con questa nuova euristica con altre euristiche notevoli (come Nearest Neighbour).


\begin{center}
        \begin{tabular}{|c|c|}
                \hline
                \textbf{Numero città} & \textbf{Numero percorsi validi} \\ % Intestazione in grassetto
                \hline % Linea orizzontale sopra della tabella
                4 & 24 \\
                \hline
                5 & 120 \\
                \hline
                6 & 720 \\
                \hline
                7 & 5,040 \\
                \hline
                8 & 40,320 \\
                \hline
                9 & 362,880 \\
                \hline
                10 & 3,628,800 \\
                \hline
                11 & 39,916,800 \\
                \hline
                12 & 479,001,600 \\
                \hline
                13 & 6,227,020,800 \\
                \hline
                14 & 87,178,291,200 \\
                \hline
                15 & 1,307,674,368,000 \\
                \hline
                16 & 20,922,789,888,000 \\
                \hline
                17 & 355,687,428,096,000 \\
                \hline
                18 & 6,402,373,705,728,000 \\
                \hline
                19 & 121,645,100,408,832,000 \\
                \hline
                20 & 2,432,902,008,176,640,000 \\
                \hline
                21 & 51,090,942,171,709,440,000 \\
                \hline
                22 & 1,124,000,727,777,607,680,000 \\
                \hline
                23 & 25,852,016,738,884,976,640,000 \\
                \hline
                24 & 620,448,401,733,239,439,360,000 \\
                \hline
                25 & 15,511,210,043,330,985,984,000,000 \\
                \hline
        \end{tabular}
        \newline
        % Aggiungi una riga per la descrizione sotto alla tabella
        Tabella 1: Numero percorsi validi in relazione al numero di città
\end{center}



%
%
%			ORGANIZZAZIONE
%
%			RINGRAZIAMENTI
%

\afterpreface
% 
% 
%			CAPITOLO 1: dshjkfg
\chapter{Introduzione}
AAA
%
%

%
%			BIBLIOGRAFIA
%
\begin{thebibliography}{00}
%
\bibitem{gotti91}
M. Gotti, I linguaggi specialistici, Firenze, La Nuova Italia, 1991.
%
\bibitem{wellek62}
R. Wellek, A. Warren, Theory of Literature , 3rd edition, New York, Harcourt, 1962.
%
\bibitem{canziani78}
A. Canziani et al., Come comunica il teatro: dal testo alla scena. Milano, Il Formichiere, 1978.
%
\bibitem{MoD67}
Ministry of Defence, Great Britain, Author and Subject Catalogues of the Naval Library, London, Ministry of Defence, HMSO, 1967.
%
\bibitem{heine23}
H. Heine, Pensieri e ghiribizzi. A cura di A. Meozzi. Lanciano, Carabba, 1923.
%
\bibitem{basso62}
L. Basso, ``Capitalismo monopolistico e strategia operaia'', Problemi del socialismo, vol. 8, n. 5, pp. 585-612, 1962.
%
\bibitem{avirovic93}
L. Avirovic, J. Dodds (a cura di), Atti del Convegno internazionale "Umberto Eco, Claudio Magris. Autori e traduttori a confronto" ( Trieste, 27-28 novembre 1989), Udine, Campanotto, 1993.
%
\bibitem{gans67}
E.L. Gans, "The Discovery of Illusion: Flaubert's Early Works, 1835-1837", unpublished Ph.D. Dissertation, Johns Hopkins University, 1967.
%
\bibitem{harrison92}
R. Harrison, Bibliography of planned languages (excluding Esperanto).  \url{http://www.vor.nu/langlab/bibliog.html}, 1992, agg. 1997.
%
\end{thebibliography}
% 
\end{document}


 
